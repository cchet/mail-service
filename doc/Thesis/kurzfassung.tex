\chapter{Kurzfassung}
Die Analyse der bestehenden Anwendung \emph{CCMail} wurden viele Fehlentscheidungen und Probleme ausgemacht, die dazu geführt haben, dass diese Anwendung neuen Anforderungen nicht mehr gerecht werden kann. Das Software-Designs sowie das implementierte Datenbankschema wurden zu lange nicht gewartet und haben sich zu lagen nicht den neuen Standards und Technologien angepasst, was dazu geführt hat, das ein Refaktorisieren einer Reimplementierung gleichzusetzen ist. Nichts desto trotz wurden im Konzept von \emph{CleverMail} viele Eigenschaften von \emph{CCMail} übernommen. Vor allem was den E-Mail-Versand selbst betrifft, der, bis auf die neuen Feature, beinahe glich verläuft. Im Konzept von \emph{CleverMail} wurden vor allem neue Möglichkeiten, die mit der Verwendung der JEE-7-Plattform zur Verfügung stehen, vorgestellt. Vor allem das Problem, der Vorlagenparameter stellt eine Herausforderung dar, da diese in vielen Bereichen von \emph{CleverMail} verwendet werden. Hierbei ist besonders auf die Konsistenz zu achten, da Änderungen an dieser Spezifikation weitreichende Folgen haben können. 
\newline
\newline
Eine weitere Herausforderung stellt die Integration in die verschiedenen Anwendungen im \emph{clevercure} Ökosystem dar, da diese einerseits alle in Java implementiert wurden jedoch auf unterschiedlichen Plattformen laufen und sich daher auch die unterstützten Technologien und Frameworks stark unterschieden. Das Konzept von \emph{CleverMail} stellt eine gute Basis dar auf die man aufbauen kann. Ein Prototyp sollte hier implementiert werden damit die vorgestellten Konzepte auf ihre Tauglichkeit getestet werden können. Abschließend sei angemerkt das anfangs nicht angenommen wurde dass sich so viele Teile von \emph{CCMail} in \emph{CleverMail} wiederfinden werden. Dies wird aber das Wechseln von \emph{CCMail} auf \emph{CleverMail} erleichtern und nur als positiv anzusehen.