\chapter{Kurzfassung}
In der Analyse der bestehenden Anwendung \emph{CCMail} wurden viele Fehlentscheidungen und Probleme identifiziert, die dazu geführt haben, dass \emph{CCMail} den neuen Anforderungen nicht mehr gerecht werden kann. Das Softwaredesign sowie das Datenbankschema wurden zu lange nicht vernachlässigt und haben sich zu lange nicht den neuen Standards und Technologien angepasst, was dazu geführt hat, das eine Umstrukturierung einer Reimplementierung gleichzusetzen wäre. Trotzdem wurden in das Konzept von \emph{CleverMail} viele Eigenschaften von \emph{CCMail} übernommen. Vor allem was den \emph{E-Mail}-Versand selbst betrifft, der bis auf die neuen Möglichkeiten, beinahe glich verläuft. Im Konzept von \emph{CleverMail} wurden vor allem neue Möglichkeiten, die mit der Verwendung der JEE-7-Plattform zur Verfügung stehen, berücksichtigt. Vor allem das Problem, der Vorlagenparameter stellt eine Herausforderung dar, da diese in vielen Bereichen von \emph{CleverMail} verwendet werden. Hierbei ist besonders auf die Konsistenz zu achten, da Änderungen an dieser Spezifikation weitreichende Folgen haben können. 
\newline
\newline
Eine weitere Herausforderung stellt die Integration von \emph{CleverMail} in die verschiedenen Anwendungen im Gesamtsystem von \emph{Clevercure} dar, da die Anwendungen einerseits alle in unterschiedlichen \emph{Java}-Versionen implementiert wurden und andererseits in verschiedenen Laufzeitumgebung betrieben werden und sich daher auch die unterstützten Technologien und Bibliotheken stark unterschieden. Das Konzept von \emph{CleverMail} stellt eine gute Basis dar. Mit einem implementierten Prototypen können die vorgestellten Konzepte auf ihre Tauglichkeit getestet werden. Abschließend sei angemerkt, dass anfangs nicht angenommen wurde, dass sich so viele Aspekte von \emph{CCMail} sich in \emph{CleverMail} wiederfinden. Dies wird aber das Wechseln von \emph{CCMail} auf \emph{CleverMail} erleichtern und ist als positiv anzusehen. Die wesentlichen Unterschiede zwischen \emph{CCMail} und \emph{CleverMail} sind:
\begin{itemize}
	\item Die Datenbank ist nicht mehr die Schnittstelle zwischen \emph{CCMail} und den anderen Anwendungen wie z.B. \emph{CleverWeb},
	\item Die Integration in andere Softwarekomponenten in anderen Laufzeitumgebungen ist jetzt möglich und
	\item Die \emph{E-Mail}-Vorlagen sind nicht mehr statisch definiert sondern können dynamisch erstellt und modifiziert werden.
\end{itemize}
\ \newline
Vor allem die Möglichkeit der Integration von \emph{CleverMail} in andere Softwarekomponenten ist hervorzuheben, da diese in \emph{CCmail} nicht unterstützt wurde, aber in \emph{CleverMail} ein zentraler Bestandteil ist. Dadurch wird sichergestellt, dass in \emph{CleverMail} alle Implementierungen gekapselt werden und von anderen Softwarekomponenten verwendet werden können.