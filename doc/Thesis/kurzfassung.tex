\chapter{Kurzfassung}
Bei der Analyse der bestehenden Anwendung \emph{CCMail} wurden nach der Meinung des Autors viele Fehlentscheidungen getroffen, die dazu geführt haben, dass diese Anwendung heute nicht mehr refaktorisiert werden kann und damit nicht mehr den neuen Anforderungen gerecht werden kann. Vor allem das Design der Hauptimplementierung mit dem Namen \emph{CCBsaicEmail} sowie die Abbildung der einzelnen E-Mail-Typen als Ableitung dieser Klasse  war zu bemängeln, da es sehr unflexibel und starr ist. Die Tatsache das die einzelnen E-Mail-Typen beinahe keine Logik beinhalten sondern lediglich die Basisklasse konfigurieren hätte besser abgebildet werden können. Ebenso wurden beim Datenbankschema Fehler gemacht wobei hier die fehlenden Constraints der Referenzen anzumerken sind, die maßgeblich die Datenkonsistenz gefährden. Auch die Entscheidung die Daten für die E-Mail-Nachrichten jedes Mal frisch aus der Datenbank zu selektieren und somit die Wiederversendbarkeit der E-Mail-Nachrichten unmöglich zu machen wurde kritisiert.
\newline
\newline
Diese Analyse hatte maßgeblichen Einfluss auf das erstellte Konzept, das einige Wege aufzeigt, wie die Reimplementierung mit den Name \emph{CleverMail} aussehen könnte und welche Technologien anwendbar wären und mit welchen Problemen und Gegebenheiten man sich auseinandersetzen muss. Diese Konzept soll als Grundlage für die Implementierung beziehungsweise die technische Konzeption herangezogen werden, bei der das endgültige Design und die Technologien definiert werden, mit der \emph{CleverMail} umgesetzt werden soll. Schlussendlich hat sich aber herausgestellt, dass obwohl viele Kritikpunkte beim Design und der Umsetzung von \emph{CCMail} gefunden wurden, das neue Konzept sehr viel konzeptionelles von \emph{CCMail} übernommen hat. Es wurden zwar neue Technologien und Ansätze vorgestellt, trotzdem lässt sich im neuen Konzept viel von \emph{CCMail} wiederfinden. Dies war für den Autor anfänglich nicht absehbar, dass es sich so entwickeln wird. 