\chapter{Einleitung}
\label{cha:einleitung}
Die vorliegende Sachlage beschäftigt sich mit der Konzeption einer neuen Mail-Anwendung, welche in weiterer Folge als \emph{CleverMail} bezeichnet wird, die eine bestehende alte Mail-Anwendung, in weiterer Folge \emph{CCMail} genannt, ersetzen soll. Dieses Konzept wird für das Unternehmen \emph{curecomp} erstellt. Einleitend wird das Unternehmen \emph{curecomp} und dessen Anwendungen vorgestellt. 
\newline
\newline
Das Unternehmen \emph{curecomp} ist ein Dienstleister im SRM-Bereich (\emph{Supplier-Relationship-Management}) und betreibt eine Softwarelösung namens \emph{clevercure}, dessen Komponenten aus den folgenden Anwendungen besteht:
\begin{itemize}
	\item\emph{CleverWeb} ist eine Web-Anwendung für den webbasierten Zugriff auf clevercure.
	\item\emph{CleverInterface} ist eine Schnittstellen-Anwendung für die Anbindung der ERP-Systeme der Kunden und Lieferanten, deren Daten mittels XML-Dateien	importiert und exportiert werden können.
	\item\emph{CleverSupport} ist eine Web-Anwendung, die zur Unterstützung der \emph{Support}-Abteilung dient.
	\item\emph{CleverDocument} ist ein Dokumentenmanagementsystem für die Verwaltung aller anfallenden Dokumente.
\end{itemize} 
\ \newline
Alle diese Anwendungen erfordern den Versand von \emph{E-Mails}, um verschiedene Systemzustände und Benachrichtigungen den BenutzerInnen mitzuteilen wie z.B.:
\begin{itemize}
	\item Fehlermeldungen,
	\item Statusänderungen bei Bestellungen (erstellt, geliefert, storniert, ...),
	\item Lieferverzugsmeldungen und
	\item Registrierung eines neuen Lieferanten.
\end{itemize}
\ \newpage
\parindent0pt{Es} sind durch die Kunden und das Unternehmen \emph{curecomp} neue Anforderungen an \emph{CCMail} gestellt worden, die sich nicht mehr in \emph{CCMail} umsetzen lassen. Dies ist begründet in dem Design und der Implementierung von \emph{CCMail}. Diese Arbeit befasst sich einerseits mit der Diskussion des Designs und der Implementierung von \emph{CCMail} und liefert andererseits ein Konzept für \emph{CleverMail}.
\newline
\newline
Vor der Erstellung dieses Konzepts, wird die bestehende Anwendung \emph{CCMail} , insbesondere deren Design und Implementierung diskutiert, damit aufgezeigt werden kann, welche Designentscheidungen und Implementierungsdetails ein Erweitern von \emph{CCMail} verhindern. In dem Konzept für \emph{CleverMail} sollen die in \emph{CCMail} gemachten Design- und Implementierungsfehler berücksichtigt werden, damit \emph{CleverMail} auch zukünftig neuen Anforderungen gewachsen ist und sich diese neue Anforderungen ohne größere Probleme und Durchführungsaufwand integrieren lassen. Zukünftige Anforderungen sind zwar schwer vorauszusagen, jedoch kann man sich bei seinen Designentscheidungen, der Wahl der verwendeten Softwaremuster und Anwendungsarchitektur auf neue Anforderungen bzw. Änderungen an der bestehenden Anwendung sehr gut vorbereiten.
\newline
\newline
Für die Konzipierung von \emph{CleverMail} wurden folgende technischen Grundvoraussetzungen definiert:
\begin{itemize}
	\item \emph{Java-JDK-8} (Java-Development-Kit in der Version 1.8),
	\item \emph{JEE-7-Platform} (Java-Enterprise-Edition Plattform in der Version 7),
	\item \emph{DB2} (Proprietäre relationale Datenbank von IBM) und
	\item \emph{Wildfly} (RedHat-Applikationsserver, früher bekannt als JBoss-AS).
\end{itemize} 
\ \newline
Über die Zeit haben sich die Anforderungen an \emph{CCMail} so drastisch geändert, dass diese nicht mehr in \emph{CCMail} integriert werden können. Wie bereits erwähnt, liegt dies vor allem am Design von \emph{CCMail}. \emph{CCMail} wurde im Jahr 2002 in \emph{Java 1.4} implementiert und hatte daher nicht die technischen Möglichkeiten, die heute zur Verfügung stehen. Zur Erinnerung: \emph{Generics} stehen erst seit der Version \emph{Java 1.5} zur Verfügung. Bis heute wurden Änderungen in \emph{CCMail} vorgenommen, die keine technologischen Weiterentwicklungen von Java berücksichtigten. Aus heutiger Sicht scheint eine Erweiterung von \emph{CCMail} alleine schon wegen dem großen technologischen Unterschied der Java-Versionen 1.4 und 1.8 sinnlos.
\newline
\newline
Technologische Weiterentwicklungen fanden zwar in in den anderen Anwendungen wie \emph{CleverWeb} und \emph{CleverInterface} statt, jedoch scheint es so, dass \emph{CCMail} hier vernachlässigt wurde, was dazu geführt hat, dass ein großer technologischer Unterschied zwischen den Anwendungen entstanden ist. Daher wurde die Entscheidung getroffen, \emph{CCMail} durch \emph{CleverMail} zu ersetzten, wobei folgender Bachelorarbeit die Grundlage dafür erarbeiten soll.
\newline
\newline
Im Kapitel \ref{cha:ccmail} wird die alte Mail-Anwendung \emph{CCMail} kritisch betrachtet und analysiert. Folgende Aspekte von\emph{CCMail} finden dabei Beachtung:
\begin{itemize}
	\item Der Systemaufbau von \emph{CCMail}.
	\item Der Prozess des \emph{E-Mail}-Versand von \emph{CCMail}.
	\item Das Software-Design von \emph{CCMail}.
	\item Die Persistenz der Daten von \emph{CCMail}.
\end{itemize}
\ \newline
Die erarbeiteten Ergebnisse werden in weitere Folge dazu verwendet um das Konzept für die neue Mail-Anwendung \emph{ClverMail} zu erstellen. 
\newline
\newline
Das Konzept für \emph{CleverMail} wird im Kapitel \ref{cha:clevermail} auf Grundlage der Betrachtungen, die im Kapitel \ref{cha:ccmail} erarbeitet wurden, erstellt. Dabei werden auch die neuen Anforderungen, die an die Mail-Anwendung gestellt wurden, berücksichtigt. Das erstellte Konzept wird Möglichkeiten für die Implementierung von \emph{CleverMail} aufzeigen. Dabei werden folgende Aspekte behandelt:
\begin{itemize}
	\item Ein möglicher Systemaufbau von \emph{CleverMail}.
	\item Die implementierenden Prozesse von \emph{CleverMail}.
	\item Die Persistenz der Daten von von \emph{CleverMail}.
\end{itemize}
\ \newline
Das erstellte Konzept für die Umsetzung von \emph{CleverMail} vor allem neue Technologien und \emph{Frameworks} verwenden. Dadurch soll \emph{CleverMail} die heute zur Verfügung stehenden Möglichkeiten bestmöglich anwenden.