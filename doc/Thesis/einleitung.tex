\chapter{Einleitung}
\label{cha:einleitung}
Das Ziel dieser Arbeit liegt in der Konzeption von \emph{CleverMail}, welche die bestehende Anwendung \emph{CCMail} ablösen soll. Bevor dieses Konzept erstellt wird, wird die bestehende Anwendung \emph{CCMail} insbesondere dessen Design und Implementierung diskutiert, damit aufgezeigt werden kann welche Designentscheidungen und Implementierungsdetails ein Erweitern von \emph{CCMail} verhindern. In dem Konzept für \emph{CleverMail} sollen die in \emph{CCMail} gemachten Design- und Implementierungsfehler berücksichtigt werden damit \emph{CleverMail} auch zukünftig neuen Anforderungen gewachsen ist und sich diese neue Anforderungen ohne größere Probleme und Refaktorisierungsaufwand integrieren lassen. Zukünftige Anforderungen sind zwar schwer vorauszusagen jedoch kann man sich bei seinen Designentscheidungen, der Wahl der verwendeten Softwaremuster und Anwendungsarchitektur auf neue Anforderungen bzw. Änderungen an der bestehenden Anwendung sehr gut vorbereiten.
\newline
\newline
Für die Konzipierung von \emph{CleverMail} wurden folgende technischen Grundvoraussetzungen definiert:
\begin{enumerate}
	\item Java-8\footnote{Java Programmiersprache in der Version 1.8}
	\item JEE-7-Platform\footnote{Java-Enterprise-Edtion Platform in Version 7}
	\item DB2\footnote{IBM proprietäre relationale Datenbank}
	\item Wildfly\footnote{RedHat Applikationsserver früher bekannt als JBOSS-AS (JBOSS-Application-Server)}
\end{enumerate}
\ \newpage
\parindent0pt{Als} Unterstützung für diese Arbeit wurden folgende literarischen Werke gewählt:
\begin{enumerate}
	\item Refactoring to patterns\cite{refactoreDatabase}
	\item Refactoring Databases\cite{refactoreToPatterns}
	\item Patterns of enterprise application architecture\cite{patternsOfEnterprise}
\end{enumerate}
\ \newline
Über die Zeit haben sich die Anforderungen an \emph{CCMail} derartig geändert, dass diese nicht mehr in \emph{CCMail} integriert werden können. Wie im Vorwort \ref{cha:vorwort} bereits erwähnt liegt dies vor allem am Design von \emph{CCMail}. \emph{CCMail} wurde im Jahre 2002 in \emph{Java 1.4} implementiert und hatte daher nicht die technischen Möglichkeiten die heute zur Verfügung stehen. Als Erinnerung, Java Generics stehen erst seit der Version \emph{Java 1.5} zur Verfügung. Ebenso sind bis heute alle Erweiterungsmöglichkeiten in \emph{CCMail} ausgeschöpft worden, wobei eine Erweiterung alleine schon wegen dem großen technologischen Unterschied der Java Versionen \emph{Java 1.4} und \emph{Java 1.8} sinnlos erscheint. Weiterentwicklungen fanden zwar in in den anderen Anwendungen wie \emph{CleverWeb} und \emph{CleverInterface} statt, jedoch scheint es so dass \emph{CCMail} hier vernachlässigt wurde, was dazu geführt hat, dass ein derartig großer technologischer Unterschied entstanden ist. 
\newline
\newline
Daher wurde die Entscheidung getroffen \emph{CCMail} durch \emph{CleverMail} zu ersetzten, wobei diese Arbeit die Grundlage dafür bieten soll.