\chapter{Einleitung}
\label{cha:einleitung}
\section{Zielsetzung}
\label{sec:zielsetzung}
Das Ziel dieser Arbeit liegt in der Konzeption von \emph{CleverMail}, welche die bestehende Anwendung \emph{CCMail} ablösen soll. Bevor dieses Konzept erstellt wird, wird die bestehende Anwendung \emph{CCMail} insbesondere dessen Design und Implementierung diskutiert damit aufgezeigt werden kann welche \newline
Designentscheidungen und Implementierungen ein Erweitern von \emph{CCMail} verhindern. Im neuen Konzept sollen die in \emph{CCMail} gemachten Fehler und Fehlentscheidungen berücksichtigt werden damit \emph{CleverMail} auch \newline
zukünftig neuen Anforderungen gewachsen ist und sich neue Anforderungen ohne größere Probleme integrieren lassen. Zukünftige Anforderungen sind zwar schwer vorauszusagen jedoch kann man sich bei seinen Designentscheidungen, der Wahl der verwendeten Softwaremuster und Anwendungsarchitektur auf neue Anforderungen bzw. Änderungen an der bestehenden Anwendung gut vorbereiten.
\newline\newline
Für die Konzipierung von \emph{CleverMail} wurden folgende technischen Grundvoraussetzungen definiert:
\begin{enumerate}
	\item Java 1.8
	\item IBM-DB2 (Datenbank)
\end{enumerate}
\ \newline
Das Konzept von \emph{CleverMail} soll vor allem auf neue Konzepte und Frameworks in Java konzentrieren, da man hier keine Rücksicht auf Rückwärtskompatibilität nehmen muss. 
\newpage
Als Unterstützung für diese Arbeit wurden folgende beiden literarischen Werke gewählt:
\begin{enumerate}
	\item Refactoring to patterns\footnote{Author: Joshua Kerhievsky, ISBN: 0-321-21335-1}
	\item Refactoring Databases\footnote{Author: Scott W.Ambler/Pramod J.Sadalage, ISBN: 0-321-29353-3} 
\end{enumerate}
\ \newline
Über die Zeit haben sich die Anforderungen an \emph{CCMail} derartig geändert, dass diese nicht mehr in \emph{CCMail} integriert werden können. Wie im Vorwort bereits erwähnt liegt dies vor allem an dem Design von \emph{CCMail}. \emph{CCMail} wurde im Jahre 2002 implementiert in \emph{Java 1.4} implementiert und hatte daher nicht die technischen Möglichkeiten die uns heute zur Verfügung stehen. Als Erinnerung, Java Generics wurden erst mit der Version \emph{Java 1.5 auch bekannt als Java 5.0} implementiert. Auch sind bis heute alle Erweiterungsmöglichkeiten in \emph{CCMail} ausgeschöpft worden, wobei eine Erweiterung alleine schon wegen dem großen Versionsunterschied zwischen \emph{Java 1.4} und \emph{Java 1.8} sinnlos erscheint. Weiterentwicklungen fanden zwar in \emph{CleverWeb} und \emph{CleverInterface} statt jedoch scheint es so, dass \emph{CCMail} hier vernachlässigt wurde, was dazu geführt hat, dass ein derartig großer technologischer Unterschied eingetreten ist. 