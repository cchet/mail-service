\chapter{Abstract}
During the analysis of the existing application \emph{CCMail}, many mistakes has been discovered, which lead to the fact that this application is not capable of meeting the new requirements. The software design and the implemented database schema weren't maintained for tool long and didn't adapt to the new standards, which lead to the fact that an refactoring is now equal to an re-implementation. Nevertheless some features of \emph{CCMail} were adapted to \emph{CleverMail}. Especially the email sending process were adapted, except for the new introduced features. The concept of \emph{CleveeMail} introduced new possibilities which are able now by using the JEE7 platform. The management of the template parameters will be a great challenge, because they are used in many aspects of the newly introduced \emph{CleverMail} application. Especially the consistency of these parameters needs to be ensured, because changes made on the template parameter specification could cause far-reaching impact on the application.
\newline
\newline
Another challenge will be the integration into the other applications part of the \emph{clevercure} ecosystem, because they are implemented in Java but run on different platforms where the supported and available technologies and frameworks will differ. The concept of \emph{CleverMail} represents a good basis which to build on. A prototype should be implemented which shall be used to test the introduced concepts for usability. Last but not least it should be mentioned that it weren't expected to find some much of \emph{CCMail} in \emph{CleverMail}. This will facilitate the move from \emph{CCMail} to \emph{CleverMail} and is considered to be a positive side effect.