\chapter{Vorwort} 	% engl. Preface
Folgende Arbeit beschäftigt sich mit der Konzeption einer Mail-Anwendung für die Firma curecomp GembH, welche in weiterer Folge als \emph{CleverMail}\newline
bezeichnet wird, die eine bestehende Mail-Anwendung, in weiterer Folge\newline
\emph{CCMail} genannt, ablösen soll. Die  Firma curecomp GembH ist ein Dienstleister im Supplier Relationship Management (SRM) und betreibt eine Softwarelösung namens \emph{clevercure}, die aus folgenden Anwendungen besteht:\newline
\begin{enumerate}
	\item\emph{CleverWeb}
	\newline
	Eine Web-Anwendung für den webbasierten Zugriff
	\item\emph{CleverInterface}
	\newline
	Eine Schnittstellen-Anwendung für die Anbindung der ERP-Systeme der Kunden und Lieferanten
	\item\emph{CleverSupport}
	\newline
	Eine Web-Anwendung, die zur Unterstützung der Support-Abteilung dient.
	\item\emph{CleverDocument}
	\newline
	Ein Dokumentenmanagementsystem für die Dokumentenverwaltung aller anfallenden Dokumente 
\end{enumerate} 
\ \newline
Beide Anwendungen erfordern den Versand von E-Mail-Nachrichten um verschiedene Systemzustände den Benutzern mitzuteilen wie z.B.:
\begin{itemize}
	\item Fehlermeldungen
	\item Statusänderungen bei Bestellungen (erstellt, geliefert, storniert, ...)
	\item Lieferverzugsmeldungen
	\item ...
\end{itemize}
\ \newline
Es wurden neue Anforderungen für \emph{CCMail}
definiert, die sich nicht mehr in \emph{CCMail} integrieren lassen. Dies ist begründet in dem Design und der Implementierung von \emph{CCMail}.
\newline\newline
Diese Arbeit wird sich einerseits mit der Diskussion des bestehenden Designs und der bestehenden Implementierung von \emph{CCMail} befassen und andererseits ein Konzept für 
\emph{CleverMail} einbringen. Das eingebrachte Konzept soll als Basis für die praktische Bachelorarbeit dienen, in der das eingebrachte Konzept umgesetzt werden soll.
