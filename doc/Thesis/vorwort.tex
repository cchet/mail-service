\chapter{Vorwort} 	% engl. Preface
\label{cha:vorwort}
Folgende Arbeit beschäftigt sich mit der Konzeption einer Mail-Anwendung, welche in weiterer Folge als \emph{CleverMail} bezeichnet wird, die eine bestehende Mail-Anwendung, in weiterer Folge \emph{CCMail} genannt, ersetzen soll. Dieses Konzept wird für die Firma \emph{curecomp GmbH} erstellt, dessen bestehende Mail-Anwendung \emph{CCMail} ersetzt werden soll. Einleitend wird die Firma \emph{curecomp GmbH} und dessen Softwareökosystem vorgestellt. 
\newline
\newline
Die Firma \emph{curecomp GmbH} ist ein Dienstleister im SRM-Bereich (Supplier-Relationship-Management) und betreibt eine Softwarelösung namens \emph{clevercure}, dessen Ökosystem aus den folgenden Anwendungen besteht:
\newline
\begin{enumerate}
	\item\emph{CleverWeb}
	\newline
	Eine Web-Anwendung für den webbasierten Zugriff auf das System
	\item\emph{CleverInterface}
	\newline
	Eine Schnittstellen-Anwendung für die Anbindung der ERP-Systeme der Kunden und Lieferanten, deren Daten mittels XML-Dateien
	\newline
	import und exportiert werden können
	\item\emph{CleverSupport}
	\newline
	Eine Web-Anwendung, die zur Unterstützung der Support-Abteilung dient
	\item\emph{CleverDocument}
	\newline
	Ein Dokumentenmanagementsystem für die Dokumentenverwaltung aller anfallenden Dokumente.
\end{enumerate} 
\ \newline
Alle diese Anwendungen im Ökosystem erfordern den Versand von E-Mail-Nachrichten um verschiedene Systemzustände und Benachrichtigungen den Benutzern mitzuteilen wie z.B.:
\begin{itemize}
	\item Fehlermeldungen
	\item Statusänderungen bei Bestellungen (erstellt, geliefert, storniert, ...)
	\item Lieferverzugsmeldungen
	\item Registrierung eines neuen Lieferanten
\end{itemize}
\ \newpage
\parindent0pt{Es} wurden neue Anforderungen an \emph{CCMail} gestellt, die sich nicht mehr in \emph{CCMail} realisieren lassen. Dies ist begründet in dem Design und der Implementierung von \emph{CCMail}. Diese Arbeit wird sich einerseits mit der Diskussion des Designs und der Implementierung von \emph{CCMail} befassen und andererseits ein Konzept für \emph{CleverMail} einbringen.
