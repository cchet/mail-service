\chapter{Das Ziel des Projekts}
\label{cha:Zielsetzung}
Das Ziel ist es, die Softwarekomponente Vorlagenmanagement für die Verwendung in \emph{CleverMail} zu implementieren, mit dem Vorlagen erstellt und verwaltet werden können. Das Vorlagenmanagement stellt einen essentiellen Teil \emph{CleverMail} dar und wird auch von mehreren Anwendungen verwendet werden. Die verschiedenen Anwendungen, die das Vorlagenmanagement verwenden, sind ebenfalls in Java implementiert, werden aber in unterschiedlichen Laufzeitumgebungen betrieben wie z.B.
\begin{itemize}
	\item Der \emph{IBM-Integration-Bus (IIB)}
	\newline
	ist ein proprietäres Produkt des Unternehmens \emph{IBM}, für \emph{XML}-Konvertierungen und den \emph{XML}-basierten Datenimport und Datenexport.
	\item Der Anwendungsserver \emph{Wildfly}
	\newline
	ist ein zertifizierter \emph{JEE-7} Applikationsserver des Unternehmens \emph{Redhat}.
\end{itemize} 
\ \newline
Die verschiedenen Anwendungen von \emph{clevercure} müssen mit geringsten Aufwand in der Lage sein Vorlagen zu verwenden und \emph{E-Mail}-Nachrichten auf Basis dieser Vorlagen zu erstellen. Dabei müssen die Abhängigkeiten der Anwendungen zum Vorlagenmanagement so gering wie möglich gehalten werden, sowie nur vorgegebene Schnittstellen verwendet werden dürfen. Wird eine \emph{E-Mail}-Nachricht von einer Anwendung auf Basis einer \emph{E-Mail}-Vorlage erstellt, so müssen dessen enthaltene Variablen beim Zeitpunkt des Erstellens der \emph{E-Mail}-Nachricht aufgelöst und die aktuellen Werte der Variablen serialisiert werden, damit die \emph{E-Mail}-Nachricht mit denselben Daten zu jedem Zeitpunkt erneut versendet werden kann. Für die Anwendungen darf nicht erkennbar sein, wie die \emph{E-Mail}-Nachrichten nach ihrer Erstellung weiter verwendet werden. Zurzeit interagieren die Anwendungen direkt mit der Datenbank anstatt von ihr abstrahiert zu sein und sind daher stark an die bestehende Anwendung \emph{CCMail} gekoppelt bzw. an dessen Datenbankmodell.
\newpage

\section{Die funktionalen Ziele}
Für das Vorlagenmanagement wurden die folgende funktionalen Anforderungen definiert, die umgesetzt werden müssen.

\subsection{Die Variablen für die Vorlagen}
Die Vorlagen werden für einen bestimmten \emph{Mail}-Typ definiert, der in einen bestimmten Kontext verwendet wird wie z.B.
\begin{itemize}
	\item Ein Benutzer wurde erstellt,
	\item eine Bestellung wurde erstellt oder
	\item ein Dokument wurde hochgeladen.
\end{itemize}
\ \newline
Für die Vorlagen, die für einen bestimmten \emph{Mail}-Typ erstellt werden, müssen Variablen zur Verfügung gestellt werden können wie z.B.
\begin{itemize}
	\item Die Variable \emph{CURRENT\_USER}
	\newline
	ist der Benutzer, der die \emph{E-Mail}-Nachricht erstellt halt.
	\item Die Variable \emph{ORDER\_NUMBER}
	\newline
	ist die Nummer der erstellten Bestellung.
\end{itemize}
Die EntwicklerInnen müssen für einen bestimmten \emph{Mail}-Typ in der Lage sein einfach Variablen zu definieren, die von den BenutzerInnen, beim Erstellen einer Vorlage für den korrespondierenden \emph{Mail-Typ}, frei verwendet werden können. Die Variablen müssen auch global definiert werden und in allen Vorlagen verwendbar sein. Die EntwicklerInnen müssen in der Lage sein die Menge der zur Verfügung stehenden Variablen zur Laufzeit aufgrund von bestimmten Zuständen verändern zu können. Die Menge der Variablen könnte z.B von Berechtigungen der BenutzerInnen abhängig sein.

\subsection{Die Mehrsprachigkeit der Variablen}
Die zur Verfügung stehenden Variablen werden durch die EntwicklerInnen statisch definiert und müssen einen Titel und eine Beschreibung zur Verfügung stellen. Der Titel und die Beschreibung der Variable müssen mehrsprachig zur Verfügung stehen, wobei als Standardsprache Englisch zu verwenden ist. Die Mehrsprachigkeit soll über \emph{Properties}-Dateien abgebildet werden.

\subsection{Die automatische Registrierung der Variablen}
Innerhalb einer \emph{CDI}-Umgebung sollen die definierten Variablen beim Start der \emph{CDI}-Umgebung automatisch gefunden und registriert werden. Die automatische Registrierung der Variablen muss mit einer \emph{CDI}-Erweiterung realisiert werden, die beim Start der \emph{CDI}-Umgebung, die Variablen findet, registriert und über die Anwendungslebensdauer persistent hält. Mit einer automatischen Registrierung der Variablen wird erreicht, dass neu definierte Variablen automatisch gefunden und registriert werden und somit nicht manuell registriert werden müssen. Ein manuelles Registrieren der Variablen birgt ein Risiko in sich, wenn Variablen vergessen werden registriert zu werden.

\subsection{Die Mehrsprachigkeit der Vorlagen}
Die Vorlagen müssen in mehreren Sprachen erstellt und verwaltet werden können, wobei eine Sprache als Standardsprache zu definieren ist und es für diese Sprache immer einen Eintrag geben muss. Auf die Standardsprache wird zurückgegriffen, wenn es für eine angeforderte Sprache keinen Eintrag gibt. Somit ist gewährleistet, dass für jede angeforderte Sprache immer eine Vorlage zur Verfügung steht. Es ist nicht erforderlich dass die Variablen über alle definierte Sprachen gleich sind. Es dürfen in einer Vorlage, die in mehreren Sprachen definiert wurde, eine unterschiedliche Anzahl oder unterschiedliche Variablen definiert sein.

\subsection{Die Persistenz der Vorlagen}
\label{sec:sub-template-variable-persistenz}
Die Vorlagen müssen innerhalb einer Datenbank persistent gehalten werden. Da das Vorlagenmanagement vorerst exklusiv für \emph{CleverMail} verwendet wird, muss die Persistenz der Vorlagen innerhalb des \emph{Mail}-DB-Schema von \emph{CleverMail} realisiert werden. Die persistenten Vorlagen müssen versionierbar sein, damit diese von anderen Entitäten referenziert werden können, ohne dass die Gefahr besteht, dass sich die referenzierte Vorlage verändert wurde.    Die Versionierung soll die Konsistenz der Vorlagen sicherstellen, sodass serialisierte Daten für eine Vorlage konsistent mit den enthaltenen Variablen sind. Vorlagen müssen explizit freigegeben werden, bevor diese verwendet dürfen. Nach einer Freigabe darf die Vorlage nicht mehr geändert werden.

\subsection{Die Verwaltung der Vorlagen über eine Webseite}
\label{sec:sub-template-management-website}
Die Vorlagen müssen über eine Webseite verwaltet werden können. Die Webseite muss mit der \emph{View}-Technologie \emph{JSF} implementiert werden. Über einen \emph{FacesConverter} soll die Vorlage von der \emph{HTML}-Repräsentation in die Repräsentation der verwendeten \emph{Template-Engine} konvertiert werden und visa versa. Der Quelltext aus Abbildung \ref{prog:html-markup-vorlage} zeigt ein \emph{HTML-Markup} einer Vorlage, wie es in Webseite  bzw. innerhalb des \emph{Editors CKEditor} verwendet wird. Die Variablen werden als \emph{HTML-Tags} repräsentiert, aus denen die originale Variable wieder hergestellt werden kann.
\newpage
\begin{program}[h]
	\caption{Beispiel eines \emph{HTML-Markup} einer Vorlage}
	\label{prog:html-markup-vorlage}
	\begin{HtmlCode}[numbers=none]
	<p>Das ist eine Variable:</p>
	<p>
		<span class="variable" title="Description" data-variable="VAR_1">
			Variable 1
		</span>
	</p>
	\end{HtmlCode}
\end{program}
\ \newline
Der Quelltext aus Abbildung \ref{prog:freemarker-markup-vorlage} zeigt das konvertierte \emph{HTML-Markup} aus Abbildung \ref{prog:html-markup-vorlage} als \emph{Freemarker}-Vorlage .
\begin{program}[h]
	\caption{Konvertiertes \emph{HTML-Markup} als \emph{Freemarker-Template}}
	\label{prog:freemarker-markup-vorlage}
	\begin{GenericCode}[numbers=none]
	<p>Das ist eine Variable:</p>
	<p>
		${module.core.VariableHolder["VAR_1"]!("Variable not found")}
	</p>	
	\end{GenericCode}
\end{program}
\ \newline
Auf der Webseite muss der \emph{Javascript} basierte \emph{Rich-Editor CKEditor} verwendet werden, weil für diesen \emph{Rich-Editor} von \emph{primefaces-extensions} eine \emph{JSF}-Integartion, in Form einer \emph{JSF}-Komponente, zur Verfügung gestellt wird. Es muss der \emph{Editor CKEditor} verwendet werden, weil eine Integration in den Lebenszyklus von \emph{JSF} notwendig ist, damit z.B. auf \emph{AJAX-Reqeust} reagiert werden kann, wie es in \emph{JSF} üblich ist.

\section{Die technischen Ziele}
\label{sec:technical-goals}
Es wurden die folgenden technischen Zile definiert.
\begin{itemize}
	\item Die Entwicklung in \emph{Java 8},
	\item die Entwicklung mit der Plattform \emph{JEE-7},
	\item die Integration in eine \emph{CDI}-Umgebung,
	\item die Integration in \emph{JSF} und
	\item die Entwicklung als eigene Softwarekomponente.
\end{itemize}
Das Vorlagenmanagement muss Schnittstellen definieren, die die Funktionalitäten des Vorlagenmanagements nach außen offenlegen, ohne dass die Anwendungen in Berüghrung mit den konkreten Implementierungen kommen.