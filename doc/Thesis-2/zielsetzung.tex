\chapter{Ziel des Projekts}
\label{cha:Zielsetzung}
Das Ziel des Projekts Vorlagenmanagement für \emph{CleverMail} ist die Entwicklung der Softwarekomponente Vorlagenmanagement für die Verwendung in \emph{CleverMail}, mit dem Vorlagen verwaltet werden können. Das Vorlagenmanagement stellt einen essentiellen Teil von \emph{CleverMail} dar und wird auch von mehreren Anwendungen innerhalb von \emph{clevercure} verwendet werden. Die verschiedenen Anwendungen, die das Vorlagenmanagement verwenden, sind ebenfalls in Java implementiert, werden aber in unterschiedlichen Laufzeitumgebungen betrieben werden wie z.B.:
\begin{itemize}
	\item \emph{IBM-Integration-Bus} (IIB) ist ein proprietäres Produkt des Unternehmens \emph{IBM}, das für die \emph{XML}-Konvertierungen und den \emph{XML}-basierten Datenimport und -export verwendet wird.
	\item \emph{WildFly} ist ein zertifizierter und frei verfügbarer \emph{JEE-7} Anwendungsserver des Unternehmens \emph{RedHat}.
\end{itemize} 
\ \newline
Die verschiedenen Anwendungen von \emph{clevercure} müssen mit möglichst wenig Aufwand in der Lage sein, Vorlagen zu verwenden und \emph{E-Mail}-Nachrichten auf Basis dieser Vorlagen zu erstellen. Dabei müssen die Abhängigkeiten der Anwendungen zum Vorlagenmanagement so gering wie möglich gehalten werden, sowie nur vorgegebene Schnittstellen verwendet werden. Wird eine \emph{E-Mail}-Nachricht von einer Anwendung auf Basis einer Vorlage erstellt, so müssen die  aktuellen Werte der Variablen der Vorlage beim Zeitpunkt des Erstellens der \emph{E-Mail}-Nachricht ermittelt und serialisiert werden, damit die \emph{E-Mail}-Nachricht mit demselben Inhalt erneut versendet werden kann. Für die Anwendungen darf nicht erkennbar sein, wie die \emph{E-Mail}-Nachrichten nach ihrer Erstellung weiter verwendet werden.
\newline
\newline
Zurzeit interagieren die Anwendungen direkt mit der Datenbank, anstatt von ihr abstrahiert zu sein und sind daher stark an die bestehende Anwendung \emph{CCMail} gekoppelt bzw. an das Datenbankschema der Anwendung \emph{CCMail}.
\newpage

\section{Funktionale Ziele}
Für das Vorlagenmanagement wurden die folgenden funktionalen Ziele definiert, die umgesetzt werden müssen.

\subsection{Variablen für die Vorlagen}
Die Vorlagen werden für einen bestimmten \emph{Mail}-Typ definiert, wobei eine Vorlage in einem bestimmten Kontext verwendet wird wie z.B.
\begin{itemize}
	\item ein(e) BenutzerIn wurde erstellt,
	\item eine Bestellung wurde erstellt oder
	\item ein Dokument wurde hochgeladen.
\end{itemize}
\ \newline
Für die Vorlagen, die für einen bestimmten \emph{Mail}-Typ erstellt werden, müssen Variablen zur Verfügung gestellt werden können wie z.B.:
\begin{itemize}
	\item Die Variable \emph{CURRENT\_USER} ist der Benutzer, der die \emph{E-Mail}-Nachricht erstellt halt.
	\item Die Variable \emph{ORDER\_NUMBER} ist die Nummer der erstellten Bestellung.
\end{itemize}
\ \newline
Die EntwicklerInnen müssen für einen bestimmten \emph{Mail}-Typ in der Lage sein, einfach Variablen zu definieren, die von den BenutzerInnen, beim Erstellen einer Vorlage für den korrespondierenden \emph{Mail}-Typ, frei verwendet werden können. Die Variablen müssen auch global definiert werden und prinzipiell in allen Vorlagen verwendbar sein. Die EntwicklerInnen müssen in der Lage sein, die Menge der zur Verfügung stehenden Variablen zur Laufzeit, aufgrund von bestimmten Zuständen, verändern zu können. Die Menge der Variablen könnte z.B. von Berechtigungen der BenutzerInnen abhängig sein.

\subsection{Mehrsprachigkeit der Variablen}
Die zur Verfügung stehenden Variablen werden durch die EntwicklerInnen statisch definiert und müssen jeweils einen Bezeichner, eine Bezeichnung und eine Beschreibung zur Verfügung stellen. Die Bezeichnung und die Beschreibung einer Variable müssen mehrsprachig zur Verfügung stehen, wobei als Standardsprache Englisch zu verwenden ist. Die Mehrsprachigkeit soll über \emph{Java-Properties}-Dateien abgebildet werden, wobei als Zeichenkodierung \emph{UTF8} zu verwenden ist, obwohl \emph{Java-Properties}-Dateien laut Spezifikation die Zeichenkodierung \emph{ISO 8859-1} verwenden müssen.

\subsection{Automatische Registrierung der Variablen}
Innerhalb einer \emph{CDI}-Umgebung sollen die definierten Variablen beim Start der \emph{CDI}-Umgebung automatisch gefunden und registriert werden. Die automatische Registrierung der Variablen muss mit einer \emph{CDI}-Erweiterung realisiert werden, die beim Start der \emph{CDI}-Umgebung die Variablen findet, registriert und über die Anwendungslebensdauer persistent hält. Mit einer automatischen Registrierung der Variablen wird erreicht, dass neu definierte Variablen automatisch gefunden und registriert werden und somit nicht manuell registriert werden müssen. Ein manuelles Registrieren der Variablen birgt das Risiko in sich, dass Variablen vergessen werden könnten.

\subsection{Mehrsprachigkeit der Vorlagen}
Die Vorlagen müssen in mehreren Sprachen erstellt und verwaltet werden können, wobei eine Sprache als Standardsprache zu definieren ist, und es für diese Sprache immer einen Eintrag geben muss. Auf die Standardsprache wird zurückgegriffen, wenn es für eine angeforderte Sprache keinen Eintrag gibt. Somit ist gewährleistet, dass für jede angeforderte Sprache immer eine Vorlage zur Verfügung steht. Es ist nicht erforderlich, dass die Menge und Position der Variablen in einer Vorlage über alle definierten Sprachen gleich sind. Es dürfen in einer Vorlage, die in mehreren Sprachen definiert wurde, eine unterschiedliche Anzahl von Variablen, unterschiedliche Variablen und unterschiedliche Positionen der Variablen definiert sein.

\subsection{Verwaltung der Vorlagen über eine Webseite}
\label{sec:sub-template-management-website}
Die Vorlagen müssen über eine Webseite verwaltet werden können. Die Webseite muss mit der \emph{View}-Technologie \emph{JSF} implementiert werden. Über einen \emph{FacesConverter} soll die Vorlage von ihrer \emph{HTML}-Repräsentation in die Repräsentation der verwendeten \emph{Template-Engine} konvertiert werden und vice versa. Der Quelltext \ref{prog:html-markup-vorlage} zeigt ein \emph{HTML-Markup} einer Vorlage, wie es in der Webseite  bzw. innerhalb des \emph{Editors CKEditor} verwendet wird. Die Variablen werden als \emph{HTML-Tags} repräsentiert, aus denen die Variablen wieder ermittelt werden können.
\begin{program}[h]
	\caption{Das \emph{HTML-Markup} einer Vorlage}
	\label{prog:html-markup-vorlage}
	\begin{HtmlCode}[numbers=none]
	<p>Das ist eine Variable:</p>
	<p>
		<span class="variable" 
		      title="Die Beschreibung der Variable" 
		      data-variable="VAR_ID">
   			Die Bezeichnung der Variable
		</span>
	</p>
	\end{HtmlCode}
\end{program}
\ \newpage
\parindent0pt Der Quelltext \ref{prog:freemarker-markup-vorlage} zeigt das konvertierte \emph{HTML-Markup} des Quelltexts \ref{prog:html-markup-vorlage} als \emph{Freemarker}-Vorlage .
\begin{program}[h]
	\caption{Das konvertierte \emph{HTML-Markup} als \emph{Freemarker}-Vorlage}
	\label{prog:freemarker-markup-vorlage}
	\begin{GenericCode}[numbers=none]
	<p>Das ist eine Variable:</p>
	<p>
		${module.core.VariableHolder["VAR_ID"]!("Nicht verfügbar")}
	</p>	
	\end{GenericCode}
\end{program}
\ \newline
%$
In der Webseite muss der \emph{JavaScript}-basierte \emph{CKEditor} [\cite{ckeditor}] verwendet werden, weil für den \emph{CKEditor} durch \emph{PrimeFaces-Extensions} [\cite{primefacesExtensions}] eine \emph{JSF}-Integration, in Form einer vollständigen \emph{JSF}-Komponente, zur Verfügung gestellt wird. Es muss auch deshalb der \emph{Editor CKEditor} verwendet werden, weil eine Integration in den Lebenszyklus von \emph{JSF} notwendig ist, damit z.B. auf \emph{AJAX}-Anfragen reagiert werden kann, wie es in \emph{JSF} üblich ist.

\section{Technische Ziele}
\label{sec:technical-goals}
Es wurden die im Folgenden aufgelisteten technischen Zeile definiert:
\begin{itemize}
	\item Die Entwicklung in Java 8,
	\item die Entwicklung mit der Plattform \emph{JEE-7},
	\item die Integration in eine \emph{CDI}-Umgebung,
	\item die Integration in \emph{JSF},
	\item die Integration in \emph{TypeScript} und
	\item die Entwicklung als eigene Softwarekomponente.
\end{itemize}
\ \newline
Das Vorlagenmanagement muss Schnittstellen definieren, welche die Funktionalität des Vorlagenmanagements nach außen offenlegen, ohne dass die Anwendungen in Berührung mit den konkreten Implementierungen kommen.