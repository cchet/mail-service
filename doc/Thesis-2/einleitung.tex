\chapter{Einleitung}
\label{cha:Einleitung}
Die vorliegende Bachelorarbeit beschäftigt sich mit der Spezifikation und der Implementierung eines Vorlagenmanagements für den in der theoretischen Bachelorarbeit konzipierten \emph{Mail-Service} namens \emph{CleverMail}. Das Vorlagenmanagement stellt einen essentiellen Teil von \emph{CleverMail} dar, mit dem sich parametrisierte \emph{E-Mail}-Vorlagen erstellen lassen. Das Vorlagenmanagement muss es den BenutzerInnen ermöglichen, einfach eigene parametrisierte \emph{E-Mail}-Vorlagen zu erstellen, die in Anwendungen, die \emph{CleverMail} nutzen, verwendet werden können, um benutzerdefinierte \emph{E-Mail}-Nachrichten zu erstellen und zu versenden. Mit dem Vorlagenmanagement ist es nicht mehr erforderlich, die \emph{E-Mail}-Vorlagen statisch zu definieren und die \emph{E-Mail}-Vorlagen können von den BenutzerInnen nach ihren Wünschen angepasst werden.
\newline
\newline
Aufgrund des Umfangs von \emph{CleverMail} wurde entschieden, sich vorerst auf das Vorlagenmanagement zu konzentrieren. Das Vorlagenmanagement wird für \emph{CleverMail} entwickelt, könnte jedoch ohne weiteres auch in anderen Anwendungen verwendet werden, sofern diese Anwendungen die technischen Voraussetzungen erfüllen. Das Vorlagenmanagement wird als eigene Softwarekomponente entwickelt und wird keine Abhängigkeiten zu Ressourcen von \emph{CleverMail} haben.

\section{Unternehmen \emph{curecomp Software Services GmbH}}
Das Vorlagenmanagement wird in Zusammenarbeit mit dem Unternehmen \emph{curecomp Software Services GmbH} erstellt. Das Unternehmen \emph{curecomp} ist ein Dienstleister im Bereich des \emph{Supplier Relationship Managements} (\emph{SRM}) und betreibt eine eigene Softwarelösung namens \emph{clevercure}. Die Softwarelösung \emph{clevercure} besteht aus den folgenden Anwendungen:
\newline
\begin{itemize}
	\item\emph{CleverWeb} ist eine Webanwendung für den webbasierten Zugriff auf \emph{clevercure}.
	\item\emph{CleverInterface} ist eine Schnittstellenanwendung für den \emph{XML}-basierten Datenimport und -export zwischen \emph{clevercure} und den \emph{ERP}-Systemen der Kunden und der Lieferanten.
	\item\emph{CleverSupport} ist eine unternehmensinterne Webanwendung zur Unterstützung der Abwicklung von \emph{Support}-Prozessen.
	\item\emph{CleverDocument} ist ein Dokumentenmanagementsystem für die Verwaltung aller anfallender Dokumente innerhalb von \emph{clevercure}.
	\item\emph{CCMail} ist die bestehende \emph{Mail}-Anwendung für den Versand der \emph{E-Mails} innerhalb von \emph{clevercure}, die durch \emph{CleverMail} abgelöst werden soll.
\end{itemize}
\ \newline
Das Vorlagenmanagement wird von den Anwendungen innerhalb von \emph{clevercure} verwendet werden, bevor \emph{CleverMail} fertiggestellt wird, da es bereits Softwarekomponenten innerhalb der Anwendungen von \emph{clevercure} gibt, die auf parametrierbare Vorlagen angewiesen sind.

\section{Vorlagenmanagement für \emph{CleverMail}}
Mit dem Vorlagenmanagement können Vorlagen von den EntwicklerInnen und BenutzerInnen definiert und parametrierbar erstellt werden. Damit können Vorlagen dynamisch zur Laufzeit erstellt, modifiziert und gelöscht werden. Es sind keine statischen Vorlagen für die \emph{E-Mail}-Nachrichten mehr nötig und alle damit verbunden Nachteile wie z.B. 
\begin{itemize}
	\item das neu Kompilieren und Einspielen bei Änderungen der Vorlagen,
	\item keine Möglichkeit für benutzerdefinierte Vorlagen oder
	\item keine Möglichkeit der Nutzung von dynamischen Parametern in den Vorlagen 
\end{itemize}
sind nicht mehr vorhanden. 
\newline
\newline
Das Vorlagenmanagement kann auch in einem anderen Kontext verwendet werden, wobei sich die vorliegende Bachelorarbeit ausschließlich mit der Verwendung des Vorlagenmanagements in \emph{CleverMail} beschäftigt. Das Vorlagenmanagement wird als eigene Softwarekomponente implementiert und die vorliegende Bachelorarbeit zeigt auf, wie sich das Vorlagenmanagement in Anwendungen, im Kontext von \emph{E-Mail}-Vorlagen, verwenden lässt. 
\newpage

\section{Rahmenbedingungen}
Das Vorlagenmanagement muss in Java 8 implementiert werden und muss die Plattform \emph{Java Enterprise Edition 7} (\emph{JEE-7}) verwenden, wobei folgende Spezifikationen Anwendung finden müssen:
\begin{itemize}
	\item \emph{Java Persistence API 2.1} (\emph{JPA 2.1}) (\emph{JSR 338}) ist die Spezifikation für die Persistenz in Java.
	\item \emph{Context and Dependency Injection 1.1} (\emph{CDI 1.1}) (\emph{JSR 346}) ist die Spezifikation für kontextabhängige Injektion innerhalb der Plattform \emph{JEE-7}.
	\item \emph{Java Server Faces 2.2} (\emph{JSF 2.2}) (\emph{JSR 344}) ist die Spezifikation der \emph{View}-Technologie in Java. 
\end{itemize}
\ \newline
Damit wird das Vorlagenmanagement mit den aktuellsten Standards und Spezifikationen implementiert. Die Funktionalität des Vorlagenmanagements muss weitestgehend ohne die Verwendung von Bibliotheken von Drittanbietern implementiert werden. Das Vorlagenmanagement muss folgende Integration zur Verfügung stellen:
\begin{itemize}
	\item Die Integration in \emph{CDI},
	\item die Integration in \emph{JSF} und
	\item die Integration in \emph{TypeScript}.
\end{itemize} 
\ \newline
Als Entwicklungsumgebung wird \emph{IntelliJ} [\cite{intelliJ}] verwendet, die eine bekannte Entwicklungsumgebung im \emph{Java}-Umfeld ist und ein Produkt des Unternehmens \emph{Jetbrains} mit Sitz in Tschechien ist. 
\newline
\newline
Als Anwendungsserver wird \emph{WildFly 10.0.0} [\cite{wildFly}], vormals \emph{JBossAS} genannt, des Unternehmens \emph{RedHat} verwendet, der ein zertifizierter \emph{JEE-7}-Server ist und somit alle benötigten Spezifikationen unterstützt. 
\newline
\newline
Es soll so weit wie möglich vermieden werden, Bibliotheken von Drittanbietern zu verwenden, außer sie sind für die Funktionalitäten des Vorlagenmanagements unerlässlich oder bieten einen essentiellen Vorteil.
