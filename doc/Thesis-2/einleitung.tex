\chapter{Einleitung}
\label{cha:Einleitung}
Die vorliegende Sachlage beschäftigt sich mit der Konzeption und Implementierung eines Vorlagen-\emph{Management} für den in der theoretischen Bachelorarbeit konzipierten \emph{Mail}-Service. Das Vorlagen-\emph{Management} stellt einen essentiellen Teil des \emph{Mail}-Service dar, mit dem sich parametrisierte \emph{E-Mail}-Vorlagen erstellen lassen. Das Vorlagen-\emph{Management} soll es den BenutzerInnen ermöglichen einfach eigene parametrisierte \emph{E-Mail}-Vorlagen zu erstellen, die in einer Anwendung, die den \emph{Mail}-Service nutzen, verwendet werden können, um benutzerspezifische \emph{E-Mail}-Nachrichten zu versenden. Mit dem Vorlagen-\emph{Management} ist es nicht mehr erforderlich die \emph{E-Mail}-Vorlagen statisch zu definieren und die \emph{E-Mail}-Vorlagen können von den Benutzerinnen nach ihren Wünschen angepasst werden.  

\section{Das Unternehmen curecomp Software Service GmbH}
Diese Arbeit wird in Zusammenarbeit mit dem Unternehmen \emph{curecomp Software Service GembH} erstellt. Das Unternehmen \emph{curecomp} ist ein ein Dienstleister im \emph{Supplier-Relationship-Management (SRM)} und betreibt eine eigene Softwarelösung namens \emph{clevercure}. Die Softwarelösung \emph{clevercure} besteht aus den folgenden Anwendungen:
\begin{itemize}
	\item\emph{CleverWeb} ist eine \emph{Web}-Anwendung für den webbasierten Zugriff auf \emph{clevercure}.
	\item\emph{CleverInterface} ist eine Schnittstellen-Anwendung für den XML-basierten Datenimport /-export zwischen clevercure und den ERP-Systemen der Kunden.
	\item\emph{CleverSupport} ist eine unternehmensinterne \emph{Web}-Anwendung für die Abwicklung von \emph{Support}-Prozessen.
	\item\emph{CleverDocument} ist ein Dokumentenmanagementsystem für die Verwaltung aller anfallender Dokumente innerhalb von \emph{clevercure}.
	\item\emph{CCMail} ist die bestehende \emph{Mail}-Anwendung für den Versand aller innerhalb \emph{clevercure} anfallender \emph{E-Mail}-Nachrichten.
\end{itemize}
\ \newline
Wie bereits in der theoretischen Bachelorarbeit behandelt, wird \emph{CCMail} von \emph{CleverMail} abgelöst werden, wobei dass in dieser Arbeit behandelte Vorlagenmanagement die Grundlage für \emph{CleverMail} darstellt. Alle Anwendung innerhalb der Softwarelösung \emph{clevercure} haben die Anforderung das \emph{E-Mail}-Vorlagen parametrisiert und benutzerdefiniert erstellt werden können. Diese Anforderung wird mit dem Vorlagenmanagement erfüllt.

\section{Das Vorlagenmanagement für den \emph{Mail}-Service}
Mit dem Vorlagenmanagement können \emph{E-Mail}-Vorlagen einerseits von den EntwicklerInnen und BenutzerInnen benutzerdefiniert und parametrisiert erstellt werden. Damit können \emph{E-Mail}-Vorlagen dynamisch auch zur Laufzeit erstellt, modifiziert und gelöscht werden. Es sind keine statischen \emph{E-Mail}-Vorlagen mehr nötig und alle damit verbunden Nachteile wie z.B. 
\begin{itemize}
	\item das neu Kompilieren und Einspielen bei Änderungen der \emph{E-Mail}-Vorlagen,
	\item keine Möglichkeit für benutzerdefinierten Vorlagen oder
	\item keine Möglichkeit der Nutzung von dynamischen Parametern in den \emph{E-Mail}-Vorlagen
\end{itemize}
\ \newline
eliminiert werden. Das Vorlagenmanagement kann auch in einem anderen Kontext verwendet werden, wobei diese Arbeit sich  ausschließlich mit der Verwendung des Vorlagenmanagement innerhalb des \emph{Mail}-Service beschäftigen wird. 
\newline
TODO: Add graphic about template management
\newline

\section{Die Rahmenbedingungen}
Das Vorlagenmanagement wird in Java in der Version 8 implementiert und wird sich an der \emph{Java-Enterprise-Edition 7 (JEE-7)} Spezifikation orientieren, wobei folgende Teilspezifikationen Anwendung finden.
\begin{itemize}
	\item \emph{JPA 2.1} ist die Spezifikation für die Persistenz.
	\item \emph{CDI 1.1} ist die Spezifikation für kontextabhängige Injektion innerhalb einer \emph{JEE7}-Umgebung.
	\item \emph{JSF 2.2} ist die Spezifikation der \emph{View}-Technologie. 
\end{itemize}
\ \newline
Damit wird das Vorlagenmanagement mit den aktuellsten Standards implementiert und wird daher für die Zukunft gut gewappnet sein. Die Funktionalität des Vorlagenmanagement wird weitestgehend ohne die Verwendung spezieller Bibliotheken implementiert, wobei folgende Integrationen zur Verfügung gestellt werden.
\begin{itemize}
	\item \emph{CDI}-Integration:
	\newline
	Innerhalb eines \emph{CDI-Containers} werden Objekte kontextabhängig zur Verfügung gestellt.
	\item \emph{JSF}-Integration:
	\newline
	Mit der \emph{View}-Technologie \emph{JSF} wird eine Webseite erstellt, über die die Vorlagen verwaltet werden können.
	\item \emph{Typescript}-Integration:
	\newline
	Mit \emph{Typescript} wird ein \emph{Plugin} für den \emph{Rich-Editor CKEditor} implementiert, welches die Variablen für eine \emph{E-Mail}-Vorlage innerhalb des \emph{CKEditors} verwaltet.
\end{itemize} 
\ \newline
Als Entwicklungsumgebung wird die \emph{IDE Intellij} verwendet, die eine bekannte Entwicklungsumgebung im \emph{Java}-Umfeld darstellt und ein Produkt des Unternehmens \emph{Jetbrains} mit Sitz in Tschechien ist. Als Applikationsserver wird \emph{Wildfly 10.x}, vormals \emph{JbossAS} genannt, des Unternehmens  \emph{Redhat} verwendet, der ein zertifizierter \emph{JEE-7}-Server ist und somit alle benötigten Spezifikationen unterstützt.
