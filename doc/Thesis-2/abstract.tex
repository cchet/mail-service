\chapter{Abstract}
This bachelor thesis is about the template management for the application \emph{CleverMail}, which is an application designed for the theoretical bachelor thesis and which will be used for the sending of emails. With the template management email messages can be managed during runtime and for multiple languages.
\newline
\newline
The implemented template management uses several technologies and languages such as \emph{CDI},  \emph{JSF} and \emph{TypeScript}. Especially the implementation in Java 8 and the possibility of the usage of the newly introduced language specific features such as Lambda functions, Method references and the Stream API made the source more readable. The integration of the template management in a CDI environment was easy to accomplish and demonstrated, that a software module can be integrated in a CDI environment, if it meets the necessary requirements. The implemented CDI extension will be easy to extend and one could use more features provided by CDI. For example, producers for variables could be registered, which could dynamically produce variables during runtime, instead of registering the variables during the start of the CDI environment, which are then immutable over the lifetime of the CDI environment.
\newline
\newline
During the development of the template management no noteworthy problems occurred, all of the predefined features and the integration were easy to implement, whereby the simplicity of the tests within a CDI environment need to be emphasized, which can be set up easy and are executable within an development environment, without the need of an application server environment.
\newline
\newline
The implementation of the CKEditor plugin has proved to be easy, because the editor is well  documented and there is already type information provided for TypeScript. The editor TinyMCE, which the plugin was supposed to be implemented in the first place, is poorly documented, which was the reason why we switched to the editor CKEditor. The implementation in TypeScript was a proper decision, because the source is more readable than the source in JavaScript. TypeScript will provide more language specific features in the future, which will make the source even more understandable, although a migration to a newer version of TypeScript is not needed for now.
