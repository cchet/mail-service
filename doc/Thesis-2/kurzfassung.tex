\chapter{Kurzfassung}
Das implementierte Vorlagenmanagement verwendet mehrere Technologien und Sprachen wie
\begin{itemize}
	\item\emph{CDI},
	\item\emph{JSF} und
	\item\emph{TypeScript}.
\end{itemize}
\ \newline
Vor allem die Implementierung in \emph{Java 8} und die Möglichkeit der Verwendung der neuen sprachspezifischen Funktionalitäten wie
\begin{itemize}
	\item\emph{Lambda}-Funktionen,
	\item Methodenreferenzen und
	\item die \emph{Stream-API} hat den Quelltext lesbarer gemacht.
\end{itemize}
\ \newline
Die Integration des Vorlagenmanagements in eine \emph{CDI}-Umgebung war einfach zu realisieren und hat aufgezeigt, dass jedes implementierte Softwaremodul in eine \emph{CDI}-Umgebung verwendet werden kann. Die implementierte \emph{CDI}-Erweiterung wird auch einfach zu erweitern sein und man könnte mehr Funktionalitäten, die in  \emph{CDI} zur Verfügung stehen, verwenden. Es könnten z.B. Erzeuger für Variablen registriert werden, die zur Laufzeit dynamisch Variablen erzeugen, anstatt dass die Variablen nur beim Start der \emph{CDI}-Umgebung registriert werden und dann über die Lebensdauer der \emph{CDI}-Umgebung  unveränderlich sind. 
\newline
\newline
Während der Entwicklung des Vorlagenmanagements sind keine erwähnenswerten Probleme aufgetreten, alle Funktionalitäten und Integrationen konnten einfach implementiert und getestet werden, wobei besonders die Einfachheit der Tests in einer \emph{CDI}-Umgebung hervorgehoben muss, die mit \emph{DeltaSpike}-Bibliotheken einfach aufgesetzt werden können und innerhalb einer Entwicklungsumgebung, ohne Applikationsserver, lauffähig sind.
\newline
\newline
Die Implementierung des \emph{CKEditor-Plugins} gestaltete sich als einfach, da dieser \emph{Editor} sehr gut dokumentiert ist und es bereits Typinformationen für \emph{TypeScript} gibt. Der \emph{Editor TinyMCE}, für den anfangs das \emph{Plugin} entwickelt werden sollte, ist sehr schlecht dokumentiert, daher wurde auf den \emph{Editor CKEditor} gewechselt. Die Implementierung in \emph{TypeScript} war die richtige Entscheidung, denn es hat die Entwicklung sehr vereinfacht, und der Quelltext ist lesbarer als der Quelltext in \emph{JavaScript}. Für die Zukunft wird \emph{TypeScript} noch mehr sprachspezifische Möglichkeiten bieten, die den Quelltext noch mehr vereinfachen werden, obwohl eine Migration auf eine neuere Version von \emph{TypeScript} zur Zeit nicht nötig ist.