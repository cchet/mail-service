\chapter{Kurzfassung}
Diese Bachelorarbeit behandelt das Vorlagenmanagement für die Anwendung \emph{CleverMail}, die in der theoretischen Bachelorarbeit konzipiert wurde und eine Anwendung ist, die zum Versand von \emph{E-Mail}-Nachrichten verwendet werden wird. Mit dem Vorlagenmanagement können Vorlagen für \emph{E-Mail}-Nachrichten zur Laufzeit und in mehreren Sprachen verwaltet werden.
\newline
\newline
Das Vorlagenmanagement verwendet mehrere Technologien und Sprachen wie \emph{CDI}, \emph{JSF} und \emph{TypeScript}. Vor allem die Implementierung in Java 8 und die Möglichkeit der Verwendung der neuen sprachspezifischen Funktionalitäten wie \emph{Lambda}-Ausdrücke, Methodenreferenzen und die \emph{Stream-API} hat den Quelltext vereinfacht.
\newline
\newline
Die Integration des Vorlagenmanagements in eine \emph{CDI}-Umgebung war einfach zu realisieren und hat gezeigt, dass ein Softwaremodul in eine \emph{CDI}-Umgebung einfach integriert werden kann, sofern es die nötigen Voraussetzungen erfüllt. Die implementierte \emph{CDI}-Erweiterung wird einfach zu erweitern sein und man könnte mehr Funktionalitäten, die in \emph{CDI} zur Verfügung stehen, verwenden. Es könnten z.B. Erzeuger für Variablen registriert werden, die zur Laufzeit dynamisch Variablen erzeugen, anstatt die Variablen nur beim Start der \emph{CDI}-Umgebung zu registrieren, welche dann über die Lebensdauer der \emph{CDI}-Umgebung  unveränderlich sind. 
\newline
\newline
Während der Entwicklung des Vorlagenmanagements sind keine erwähnenswerten Probleme aufgetreten, alle Funktionalitäten und die Integration konnten einfach implementiert und getestet werden, wobei besonders die Einfachheit der Tests in einer \emph{CDI}-Umgebung hervorgehoben werden muss, die mit den verwendeten \emph{DeltaSpike}-Bibliotheken einfach aufgesetzt werden können und innerhalb einer Entwicklungsumgebung, ohne Anwendungsserver, lauffähig sind.
\newline
\newline
Die Implementierung des \emph{CKEditor-Plugins} gestaltete sich einfach, da dieser \emph{Editor} gut dokumentiert ist und es bereits Typinformationen für \emph{TypeScript} gibt. Der \emph{Editor TinyMCE}, für den anfangs das \emph{Plugin} entwickelt werden sollte, ist hingegen schlecht dokumentiert, daher wurde auf den \emph{Editor CKEditor} gewechselt. Die Implementierung in \emph{TypeScript} war die richtige Entscheidung, denn es hat die Entwicklung vereinfacht, und der Quelltext ist lesbarer als der Quelltext in \emph{JavaScript}. Für die Zukunft wird \emph{TypeScript} weitere sprachspezifische Möglichkeiten bieten, die den Quelltext noch mehr vereinfachen werden, obwohl eine Migration auf eine neuere Version von \emph{TypeScript} zur Zeit nicht nötig ist.